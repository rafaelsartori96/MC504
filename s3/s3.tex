% Rafael Sartori M. dos Santos, 186154
\documentclass[brazilian,a4paper]{article}

% Título
\title{Atividade S3}
\author{Rafael Sartori M. Santos, 186154}
\date{27 de março de 2019}

% Configuração do documento
\setlength{\parskip}{3pt}
\usepackage[utf8]{inputenc} % tipo de documento UTF-8
\usepackage{mathtools} % permitir expressões matemáticas
\usepackage{breqn} % equações quebradas em várias linhas automaticamente
\usepackage{babel} % configuração da lingua portuguesa
\usepackage{caption} % para legenda de tabelas e figuras
\usepackage[
    pdfauthor={Rafael Sartori M. Santos},
    pdftitle={Atividade S3},
    pdfproducer={LaTeX (texlive) com hyperref},
    hidelinks
]{hyperref} % para links externos (href)
\usepackage{cleveref} % para referenciar tabelas e figuras melhor
\usepackage{indentfirst} % indentação de todo primeiro parágrafo
%\usepackage{graphicx} % para adicionar imagens
%\graphicspath{{../imgs_in/}{../imgs_out/}} % atalho para o caminho das imagens
%\usepackage{float} % para fixar posição de imagens
%\usepackage{subcaption} % para imagens ficarem lado a lado
% Usamos geometry pois dá mais espaço que fullpage
\usepackage{geometry} % alterar geometria do papel
\geometry{a4paper,left=1.7cm,right=1.7cm,top=1cm,bottom=2.0cm} % menor margem
%\usepackage{fullpage} % utilizamos uma versão com menos espaçamento nas bordas
%\usepackage{verbatim} % pacote para incluir arquivos em verbatim
%\usepackage{mdframed} % para enquadrar coisas
\usepackage[bitstream-charter]{mathdesign} % Mudamos a fonte para Charter BT
\usepackage[T1]{fontenc} % Mudamos a fonte para Charter BT

% Início do documento
\begin{document}

\maketitle

\section*{Avaliação}

O problema proposto sugere avaliar a importância da segurança na interação entre aplicativos em \textit{userspace} e o \textit{kernel} do sistema operacional.

\section*{Enunciado do problema}

O que é necessário ser feito para proteger o \textit{kernel} do sistema durante chamadas do sistema (\textit{syscalls})? Por que cada alteração foi necessária?

\section*{Resposta}

Como nas trocas de modos de operações, vista com maior profundidade na arquitetura x86, \textit{syscalls} também precisam de cuidado para não permitir ações maliciosas por parte de aplicativos, de rigor para negar acessos inválidos e de resiliência para que qualquer erro não cause a quebra do sistema.

Os principais cuidados são:
\begin{itemize}
    \item \textbf{Tratamento dos argumentos:} como as \textit{syscalls} são chamadas a partir do \textit{userspace}, é necessário buscar na memória do aplicativo (em modo usuário) os argumentos das chamadas. Por isso, é necessário copiar para a memória do \textit{kernel} para impedir que haja modificações em etapas posteriores à verificação dos argumentos, impedindo ataques ao \textit{kernel} em formato de alterações de valores de argumento.
    \item \textbf{Verificação dos argumentos:} com os argumentos copiados para a memória protegida do \textit{kernel}, verifica-se quanto a validade do pedido, por exemplo: o caminho do arquivo é uma \textit{string} válida? o aplicativo está tentando abrir algum arquivo que exista? o aplicativo possui permissão para escrita no arquivo? o aplicativo pode executar mais um \textit{fork}? o aplicativo pode requisitar mais memória? São questões importantes para impedir o abuso por parte de aplicações: mesmo se não parar o programa, o sistema permanecerá responsivo para que o usuário o faça? A verificação precisa assegurar isso.
    \item \textbf{Tratamento da resposta:} a resposta da \textit{system call} não pode permanecer na memória protegida do \textit{kernel}, é necessário copiá-la novamente à memória do aplicativo para torná-la acessível. É necessário verificar se as fronteiras para escrever a resposta são válidas, notando mudanças enquanto a \textit{syscall} era resolvida, senão o \textit{kernel} poderia quebrar escrevendo em área que não é mais do aplicativo devido ação maliciosa alterando posições de memória.
\end{itemize}

\end{document}
