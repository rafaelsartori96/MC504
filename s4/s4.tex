% Rafael Sartori M. dos Santos, 186154
\documentclass[brazilian,a4paper]{article}

% Título
\title{Atividade S4}
\author{Rafael Sartori M. Santos, 186154}
\date{27 de março de 2020}

% Configuração do documento
\setlength{\parskip}{3pt}
\usepackage[utf8]{inputenc} % tipo de documento UTF-8
\usepackage{mathtools} % permitir expressões matemáticas
\usepackage{breqn} % equações quebradas em várias linhas automaticamente
\usepackage{babel} % configuração da lingua portuguesa
\usepackage{caption} % para legenda de tabelas e figuras
\usepackage[
    pdfauthor={Rafael Sartori M. Santos},
    pdftitle={Atividade S4},
    pdfproducer={LaTeX (texlive) com hyperref},
    hidelinks
]{hyperref} % para links externos (href)
\usepackage{cleveref} % para referenciar tabelas e figuras melhor
\usepackage{indentfirst} % indentação de todo primeiro parágrafo
%\usepackage{graphicx} % para adicionar imagens
%\graphicspath{{../imgs_in/}{../imgs_out/}} % atalho para o caminho das imagens
%\usepackage{float} % para fixar posição de imagens
%\usepackage{subcaption} % para imagens ficarem lado a lado
% Usamos geometry pois dá mais espaço que fullpage
\usepackage{geometry} % alterar geometria do papel
\geometry{a4paper,left=1.7cm,right=1.7cm,top=1cm,bottom=2.0cm} % menor margem
%\usepackage{fullpage} % utilizamos uma versão com menos espaçamento nas bordas
%\usepackage{verbatim} % pacote para incluir arquivos em verbatim
%\usepackage{mdframed} % para enquadrar coisas
\usepackage[bitstream-charter]{mathdesign} % Mudamos a fonte para Charter BT
\usepackage[T1]{fontenc} % Mudamos a fonte para Charter BT

% Início do documento
\begin{document}

\maketitle

\section*{Avaliação}

Avaliar como o \textit{kernel} monolítico funciona com \textit{drivers}, que fazem muito uso de chamadas de sistema.

\section*{Enunciado do problema}

Como os \textit{drivers} em modo usuário podem interagir de forma eficiente com o \textit{hardware} do sistema? Que recursos os \textit{drivers} utilizam para isso?

\section*{Resposta}

Utilizando \textit{drivers} em modo usuário, além de permitir \textit{debugging} e aumentar a resiliência do sistema, permite diminuir a quantidade de chamadas para sistema através de \textit{queueing}, \textit{caching} e \textit{prefetching}:
\begin{itemize}
    \item \textbf{\textit{Queueing:}} apesar de diminuir a média do tempo de resposta, diminui o número de chamadas de sistema, para fazê-las de forma conjunta.
    \item \textbf{\textit{Caching:}} as escritas em \textit{cache} podem aumentar muito a velocidade de um disco em casos de grande número de pequenas escritas ou pequeno número de grandes escritas (torna a transferência muito rápida quando o tamanho é menor que o tamanho da \textit{cache}).
    \item \textbf{\textit{Prefetching:}} em casos de leituras, \textit{prefetching} pode tornar o uso de I/O imperceptível em algumas situações, como se todo o arquivo (mesmo que enorme) estivesse em memória. Mesmo que gastando algumas requisições de I/O por conta da especulação, o benefício é grande, como ocorre nos diversos níveis de \textit{caches} de CPU.
\end{itemize}

\end{document}
